\section{Related Work}
\label{sec:related_work}
% Sybil attacks.
The concept of a \emph{Sybil attack} was coined by Douceur in
2002~\cite{Douceur2002a}.  Using formal analysis, Douceur showed that in the
absence of a central verification authority, Sybil attacks are always possible,
but can be limited in scope by making assumptions on the attacker's available
resources.

% Sybil work focusing on PoW and social networks.
Subsequent work then focused on identifying resources that are hard to scale for
attackers, often concentrating on proof-of-work
systems~\cite{Li2012a,Borisov2006a} and social network
edges~\cite{Danezis2009a,Yu2006a}.

\cite{Wolchok2010a}

Related to the Tor network, Margolin and Levine~\cite{Margolin2008a} evaluated
a recurring fee onion routing protocol in which network participants pay a
small amount of money for circuit creation.  This should make it prohibitively
expensive to run Sybil attacks.

Bauer's work caused Tor to limit number of relays per IP
address~\cite{Bauer2007a}.
