\section{Related work}
\label{sec:related_work}
% Why previously proposed solutions don't work.
In his seminal 2002 paper, Douceur showed that only a \emph{central authority}
that verifies new nodes as they join the distributed system is guaranteed to
prevent Sybils~\cite{Douceur2002a}.  This approach conflicts with Tor's design
philosophy that seeks to eliminate central points of control and distribute
trust.  In addition, a major factor contributing to Tor's relay growth is the
low barrier of entry, allowing operators to set up relays quickly and
anonymously.  An identity-verifying authority would raise that barrier, alienate
privacy-conscious relay operators, and impede Tor's growth.  Barring a central
authority, researchers have proposed techniques that leverage a resource that is
difficult for an attacker to scale.  Two categories of Sybil-resistant schemes
turned out to be particularly popular, schemes that build on \emph{social
constraints} and schemes that build on \emph{computational constraints}.  For a
broad overview of alternative Sybil defenses, refer to Levine et
al.~\cite{Levine2006a}.

Social constraints rely on the assumption that it is difficult for an attacker
to form trust relationships with honest users, e.g., befriend many unknown
people on online social networks.  Past work leveraged this assumption in
systems such as SybilGuard~\cite{Yu2006a}, SybilLimit~\cite{Yu2008a}, and
SybilInfer~\cite{Danezis2009a}.  Unfortunately, social graph-based defenses
do not work in our setting because there is no existing trust relationship
between relay operators.\footnote{Relay operators can express in their
configuration that their relays are run by the same operator, using the
\texttt{MyFamily} option, but this denotes an \emph{intra}-person and not an
\emph{inter}-person trust relationship.} Note that we could create such a
relationship by, e.g., linking relays to their operator's social networking
account, or by creating a ``relay operator web of trust,'' but again, we
believe that such an effort would alienate relay operators and receive limited
adoption.

Orthogonal to social constraints, computational resource constraints guarantee
that an attacker that seeks to operate 100 Sybils needs 100 times the
computational resources she would have needed for a single virtual identity.
Borisov~\cite{Borisov2006a} and Li et al.~\cite{Li2012a} used computational
puzzles for that purpose.  Computational constraints work well in distributed
systems where the cost of joining the network is low.  For example, to use
BitTorrent, a user only needs to run a lightweight client.  However, this is not
the case in Tor because relay operations require constant use of bandwidth and
CPU.  Unlike in many other distributed systems, it is impossible to run 100 Tor
relays while not spending the resources for 100 relays.  In a way, computational
constraints are inherent to running a relay.

In summary, we believe that existing Sybil defences do not work well when
applied to the Tor network; its destinctive features call for special solutions.
