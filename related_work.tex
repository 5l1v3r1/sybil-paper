\section{Related work}
\label{sec:related_work}
% Why previously proposed solutions don't work.
In his seminal 2002 paper, Douceur showed that only a \emph{central authority}
that verifies new nodes as they join the distributed system is guaranteed to
prevent Sybils~\cite{Douceur2002a}.  This approach conflicts with Tor's design
philosophy that seeks to distribute trust and eliminate central points of
control.  In addition, a major factor contributing to Tor's network growth is
the low barrier of entry, allowing operators to set up relays both quickly and
anonymously.  An identity-verifying authority would raise that barrier, alienate
privacy-conscious relay operators, and impede Tor's growth.  Barring a central
authority, researchers have proposed techniques that leverage a resource that is
difficult for an attacker to scale.  Two categories of Sybil-resistant schemes
turned out to be particularly popular, schemes that build on \emph{social
constraints} and schemes that build on \emph{computational constraints}.  For a
broad overview of alternative Sybil defenses, refer to Levine et
al.~\cite{Levine2006a}.

Social constraints rely on the assumption that it is difficult for an attacker
to form trust relationships with honest users, e.g., befriend many unknown
people on online social networks.  Past work leveraged this assumption in
systems such as SybilGuard~\cite{Yu2006a}, SybilLimit~\cite{Yu2008a}, and
SybilInfer~\cite{Danezis2009a}.  Unfortunately, social graph-based defenses
do not work in our setting because there is no existing trust relationship
between relay operators.\footnote{Relay operators can express in their
configuration that their relays are run by the same operator, using the
\texttt{MyFamily} option, but this denotes an \emph{intra}-person and not an
\emph{inter}-person trust relationship.} Note that we could create such a
relationship by, e.g., linking relays to their operator's social networking
account, or by creating a ``relay operator web of trust,'' but again, we
believe that such an effort would alienate relay operators and receive limited
adoption.

Orthogonal to social constraints, computational resource constraints guarantee
that an attacker that seeks to operate 100 Sybils needs 100 times the
computational resources she would have needed for a single virtual identity.
Both Borisov~\cite{Borisov2006a} and Li et al.~\cite{Li2012a} used computational
puzzles for that purpose.  Computational constraints work well in distributed
systems where the cost of joining the network is low.  For example, a
lightweight client is enough to use BitTorrent, allowing even low-end consumer
devices to participate.  However, this is not the case in Tor because relay
operations require constant use of bandwidth and CPU.  Unlike in many other
distributed systems, it is impossible to run 100 Tor relays while not spending
the resources for 100 relays.  In a way, computational constraints are inherent
to running a relay.

Sybil attacks are a well-known problem to The Tor Project, which is reflected in
a number of both implicit and explicit Sybil defenses that are in place as of
February 2016.  First, directory authorities---the ``gatekeepers'' of the Tor
network---accept at most two relays per IP address to prevent low-resource Sybil
attacks~\cite{Bauer2007a,Bauer2007b}.  Similarly, Tor's path selection algorithm
states that Tor clients never select two relays in the same /16
network~\cite{path-spec}.  Second, directory authorities automatically assign
flags to relays, indicating their status and quality of service.  The Tor
Project has recently increased the minimal time until relays obtain the
\texttt{Stable} flag (seven days) and the \texttt{HSDir} flag (96 hours).  This
change increases the cost of Sybil attacks and gives Tor developers more time to
discover and block suspicious relays before they get in a position to run an
attack.  Finally, the operation of a Tor relay causes recurring costs---most
notably bandwidth and electricity---which can further restrain an adversary.

In summary, we believe that existing Sybil defences do not work well when
applied to the Tor network; its destinctive features call for special solutions.
