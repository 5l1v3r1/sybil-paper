\section{Related work}
\label{sec:related_work}
% Why previously proposed solutions don't work.
It is tempting to look for defenses in the substantial amount of previous work
that proposed Sybil mitigation techniques.  In his seminal 2002
paper~\cite{Douceur2002a}, Douceur formally showed that the only method
guaranteed to keep out Sybils is a \emph{central authority} that verifies new
nodes as they join the distributed system.  This approach conflicts with Tor's
design philosophy because it seeks to eliminate central points of control.  In
addition, a major factor contributing to Tor's growth is the low barrier of
entry, allowing operators to set up relays quickly and anonymously.  A
controlling authority would raise that barrier might thwart Tor's growth.
Barring a central authority, researchers have proposed techniques that build on
a resource that is difficult for an attacker to scale.  Two categories of
Sybil-resistant schemes turned out to be particularly popular: schemes that
build on \emph{social} and \emph{resource} constraints.  

Social constraints are based on the insight that it is difficult for an attacker
to build trust relationships with honest users.  Past work exploited this fact
in systems such as SybilGuard~\cite{Yu2006a}, SybilLimit~\cite{Yu2008a}, and
SybilInfer~\cite{Danezis2009a}.  Social graph-based defenses don't work in our
setting because there is no trust relationship between Tor relays.\footnote{To
clarify, relay operators are encouraged to set the \texttt{MyFamily} option
for relays under their control, but this type of trust relationship does not
span to relays run by others.}

Computational resource constraints can be useful if an attacker that seeks to
operate 100 Sybils suddenly needs 100 times the computational resources she
would need for a single virtual identity.  Previous work used computational
puzzles for that purpose~\cite{Borisov2006a,Li2012a}.  However, requiring relay
operators to complete proof-of-works is pointless as running a relay already
involves computational work, i.e., relaying data.  Still, Margolin and
Levine~\cite{Margolin2008a} evaluated a recurring fee onion routing protocol in
which network participants pay a small amount of money for circuit creation.

In parallel to Sybil prevention, research has focused on characterizing
real-world Sybils.  Wang and Kangasharju uncovered a Sybil attack in
BitTorrent's distributed hash table~\cite{Wang2012a}.  Thomas, Grier, and Paxson
found several thousand Sybil accounts on Twitter to dilute political
speech~\cite{Thomas2012a}, and much work has focused on detecting Sybils that
are used to send spam~\cite{Gao2010a}.

For a broader overview of Sybil defenses, refer to Levine, Shields, and
Margolin's technical report~\cite{Levine2006a}.
