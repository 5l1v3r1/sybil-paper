\section{Conclusion}
\label{sec:conclusion}
We have presented \sys, a novel system that employs diverse analysis techniques
to expose Sybils in the Tor network.  Equipped with this novel tool, we set out
to analyze The Tor Project's archived network data for signs of Sybil relays.
\Sys uncovered several Sybil groups, helping us gain new insight into real-world
Sybil attacks.  We found that (\emph{i}) Sybil relays are frequently configured
very similarly, and join and leave the network simultaneously; (\emph{ii})
attackers differ greatly in their technical sophistication; and (\emph{iii}) our
techniques are not only useful for spotting Sybils, but turn out to be a handy
analytics tool to monitor and better understand the Tor network.  Given the lack
of a central identity-verifying authority, it is always possible for
well-executed Sybil attacks to stay under our radar, but we found that a
complementary set of techniques can go a long way towards finding malicious
Sybils, making the Tor network more secure and trustworthy for its users.

% Code, data, and other resources for our paper are available online at
% \url{https://nymity.ch/sybilhunting/}.
