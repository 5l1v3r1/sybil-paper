\section{Conclusion}
\label{sec:conclusion}
In this paper, we worked towards finding and characterizing Sybils in the Tor
network.  We first invested significant engineering effort into the development
of sybilhunter, a command line tool to find and analyze Sybil groups.

Equipped with this new tool, we set out to analyze The Tor Project's network
data---archived as well as online data---for signs of Sybil relays.  We
uncovered several Sybil groups and gained new insight into real-world Sybil
attacks.  We found that 1) Sybil relays frequently look alike in their
appearance and behavior, 2) Sybil-running attackers differ greatly in their
technical sophistication, and 3) determining the root cause for Sybils is
challenging because it is often difficult to distinguish between benign and
malicious Sybils.

Given the lack of a central identity-verifying authority, it is always possible
for a well-executed Sybil attack to stay under our radar, but we found that a
simple set of tools and techniques can go a long way towards finding malicious
Sybils, thus making the Tor network more secure and trustworthy for its users.

Code, data, and other resources for our paper are available online at
\url{https://nymity.ch/sybilhunting/}.
