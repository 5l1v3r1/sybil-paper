\section{Introduction}
\label{sec:introduction}

% Short introduction to Sybil attacks.
Distributed systems are believed to be more robust than their centralized
counterparts against attacks such as coercion, denial of service, and
surveillance.  The Tor network is no exception. 
However, the absence of a central, controlling authority facilitates
\emph{Sybil attacks}.  Sybil attacks---introduced by Douceur in
2002~\cite{Douceur2002a}---are defined as one physical entity controlling many
virtual identities in a distributed system.  The attack is problematic because
many distributed systems make assumptions on the maximum number of malicious
nodes they can endure.

% How do Sybil attacks relate to Tor? -- DHT manipulation.
An attacker's motivation for performing a Sybil attack differs from system to
system.  In the case of Tor, attackers are frequently motivated by manipulating
Tor's distributed hash table (DHT)~\cite{rendspec} that is used to implement
onion services.  Tor clients need an onion service descriptor to learn how to
connect to a given onion service such as \url{3g2upl4pq6kufc4m.onion}, the
DuckDuckGo search engine.  These descriptors are stored in Tor's DHT.  The DHT
implementation suffers from a vulnerability that allows an attacker to predict
at which position in the DHT a descriptor will be stored.  The attacker can then
change its fingerprints to become responsible for a given onion
service~\cite{Biryukov2013a}.\footnote{This is done by brute-forcing the RSA key
pair of a relay until its fingerprint is reasonably close to the descriptor's
position in the DHT.}  Six Sybil relays are sufficient to become the sole
responsible party for an onion service, allowing the attacker to monitor (still
anonymous) clients requests, or refuse to serve the onion service, effectively
taking it down.

% How do Sybil attacks relate to Tor? -- Increased traffic exposure.
In addition to DHT manipulation, Sybils in Tor allow an attacker to increase its
exposure to traffic.  Generally, running $n$ relays equates to running a single
relay that provides as much bandwidth as all $n$ relays.  But at some point, an
attacker will have to scale horizontally because there are limits to how much
traffic can be relayed by a single machine.  As an attacker's bandwidth grows,
it is increasingly able to:
\begin{description}
	\item[Manipulate and sniff exit traffic] to steal credentials, break into
		TLS connections, or inject data~\cite{Winter2014a}.
	\item[End-to-end correlate] traffic by running entry guards as well as exit
		relays~\cite{Johnson2013a}.  Note that a network-level adversary such as
		an ISP or a government might only need to run one of both.
	\item[Harvest bridge addresses] by running middle nodes and isolating the
		IP addresses of incoming Tor connections that don't originate from
		(publicly known) guard relays~\cite{Ling2012a}.
	\item[Website fingerprint] connections on guard relays to learn what
		websites Tor users are connecting to~\cite{Juarez2014a}.
\end{description}

% Sybils can be a side effect.
Instead of being a direct means to an end, Sybil relays can be a \emph{side
effect} of another issue.  In Section~\ref{sec:sybil_groups}, we provide
evidence for what appears to be a botnet whose zombies are running Tor relays.
Similarly, in August 2013, attackers infected computers with malware that used a
Tor onion service as command and control server~\cite{Hopper2014a}.  In this
case, however, the infected machines were Tor clients, and not relays.

% Existing Sybil defense mechanisms.
As of November 2015, the Tor network employs a number of implicit and explicit
Sybil defenses.  First, directory authorities accept at most two relays per IP
address~\cite{Bauer2007b} to prevent low-resource Sybil
attacks~\cite{Bauer2007a}.  Second, The Tor Project has gradually increased the
minimal time until relays obtain the Stable (seven days) and the HSDir (96
hours) flag to increase the cost of Sybil relays and to give Tor developers more
time to discover suspicious relays before they are able to run an attack.
Finally, operating a Tor relay causes recurring resource costs---most notably
bandwidth and CPU---which can further limit an adversary.

% Why previously proposed solutions don't work.
It is tempting to look for strong defenses in the substantial amount of previous
work that proposed Sybil mitigation techniques.  In his seminal 2002
paper~\cite{Douceur2002a}, Douceur formally showed that the only method
guaranteed to keep out Sybils is a \emph{central authority} that verifies new
nodes as they join the distributed system.  This approach conflicts with Tor's
design philosophy because it seeks to eliminate central points of control.  In
addition, a major factor contributing to Tor's growth is the low barrier of
entry, allowing operators to set up relays quickly and anonymously.  A
controlling authority would raise that barrier.  Barring a central authority,
researchers have proposed techniques that build on a resource that is difficult
for an attacker to scale.  Past work has focused extensively on
\emph{computational resources} and \emph{edges in social graphs}.

Computational resources can be useful if an attacker that seeks to
operate 100 Sybils suddenly needs 100 times the computational resources she
would need for a single virtual identity.  Social graph-based Sybil detection
techniques assume that it is difficult for Sybils to establish a large number of
connections with non-Sybils.  Unfortunately, both resources do not apply to the
Tor network. Requiring relay operators to complete proof-of-works is pointless
as running a relay already involves computational work.  Social graph-based
defenses do not apply because there is no trust relationship between Tor
relays.\footnote{To clarify, relay operators are encouraged to set the
\texttt{MyFamily} option for relays under their control, but this type of
trust relationship does not span to relays run by others.}

% Our contributions.
Given the nonapplicability of existing approaches, we focus on techniques to
find Sybils in the Tor network.  We implemented these techniques in a tool,
sybilhunter, which we then use to analyze historical network data, dating back
to as early as 2007, to discover past attacks and anomalies.  Finally, we
extensively characterize the Sybil groups we discover.  To sum up, we make
the following contributions:
\begin{itemize}
	\item We consider Sybils in Tor and show how this problem differs from
		related work.
	\item We design and implement a platform to analyze past and future Tor
		data.  Our approach does not require any modifications to the Tor
		network.
	\item We characterize past incidents and publish a dataset for future
		research.
\end{itemize}

% Structure of the paper.
The rest of this paper is structured as follows.  We begin by discussing
related work in Section~\ref{sec:related_work}.  Section~\ref{sec:design}
presents the design of our analysis tools, which is then followed by
experimental results in Section~\ref{sec:results}.  We discuss our results in
Section~\ref{sec:discussion} and conclude the paper in
Section~\ref{sec:conclusion}.
