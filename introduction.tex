\section{Introduction}
\label{sec:introduction}
In a Sybil attack, an attacker controls many virtual identities to obtain
disproportionately large influence in a network.  These attacks take many
shapes, such as sockpuppets hijacking online discourse~\cite{Thomas2012a}; the
manipulation of BitTorrent's distributed hash table~\cite{Wang2012a}; and, most
relevant to our work, relays in the Tor network that seek to deanonymize
users~\cite{cmucert}.  Preventing Sybil attacks is challenging, as shown in
Douceur's seminal 2002 paper that argues that Sybil attacks are always possible
in the absence of a central authority~\cite{Douceur2002a}.  In this work, we
focus on Sybils in the Tor network, i.e., Tor relays---virtual identities---that
are controlled by a single operator.  But what harm can Sybils do in Tor?

The effectiveness of many attacks on Tor depends on how large a fraction of the
network's traffic---the consensus weight---an attacker can observe.  As the
attacker's consensus weight grows, she is increasingly able to launch the
following attacks.

\begin{description}
	\item[Exit traffic tampering:] A Tor user's traffic traverses exit relays,
		the last hop in a Tor circuit, when leaving the Tor network.
		Controlling exit relays, an attacker can sniff traffic to collect
		unencrypted credentials, break into TLS-protected connections, or inject
		malicious content~\cite{Winter2014a}.
	\item[Website fingerprinting:] Tor's encryption prevents guard relays (the
		first hop in a Tor circuit) from learning their user's online activity.
		Ignoring the encrypted payload, an attacker can still take advantage of
		flow information such as packet lengths and timings to infer what web
		site its users are connecting to~\cite{Juarez2014a}.
	\item[Bridge address harvesting:] Users behind censorship firewalls use
		bridges---basically non-public Tor relays---as hidden stepping stones
		into the Tor network.  It is important that censors cannot obtain all
		bridge addresses, which is why bridge distribution is rate-limited.  An
		attacker can harvest bridge addresses by running a middle relay and
		looking for incoming connections that do not originate from
		publicly-known guard relays~\cite{Ling2012a}.
	\item[End-to-end correlation:] By running both entry guards and exit relays,
		an attacker can use timing analysis to link a Tor user's activity to her
		identity, e.g., learn that Alice is visiting Facebook.  For this attack,
		an attacker must run at least two Tor relays or ~\cite{Johnson2013a}.
\end{description}

An attacker can increase her consensus weight by configuring her relay to
forward more traffic.  However, the capacity of a single relay is limited by its
link bandwidth and, because of the computational cost of cryptography, by CPU.
Once an adversary reaches this limit, she has to scale horizontally, i.e., add
more Sybil relays to the network.

In addition to the above attacks, an adversary needs Sybil relays to manipulate
onion services, which are TCP servers whose IP address is hidden by Tor.  In the
current onion service protocol, six Sybil relays are enough to take offline
an onion service because of a weakness in the design of the distributed hash
table (DHT) that powers onion services~\cite{Biryukov2013a}.  Finally, instead
of being a direct means to an end, Sybil relays can be a \emph{side effect} of
another issue.  In Section~\ref{sec:sybil_groups}, we provide evidence for what
appears to be botnets whose zombies are running Tor relays, perhaps because of a
misguided attempt to help the Tor network grow.  The zombie owners are likely
unaware of the relays they are running.

Sybil attacks are a well-known problem to The Tor Project, which is reflected in
a number of both implicit and explicit Sybil defenses that are in place as of
February 2016.  First, directory authorities---the ``gatekeepers'' of the Tor
network---accept at most two relays per IP address~\cite{Bauer2007b} to prevent
low-resource Sybil attacks~\cite{Bauer2007a}.  Similarly, Tor's path selection
algorithm~\cite{path-spec} incorporates heuristics so Tor clients avoid relays
that could be controlled by the same operator, e.g., Tor clients never select
two relays in the same /16 network.  Second, directory authorities assign flags
to relays, indicating their status and quality of service.  The Tor Project has
recently increased the minimal time until relays obtain the \texttt{Stable} flag
(seven days) and the \texttt{HSDir} flag (96 hours).  This change increases the
cost of Sybil attacks and gives Tor developers more time to discover and block
suspicious relays before they get in a position to run an attack.  Finally,
operating a Tor relay causes recurring costs---most notably bandwidth and
electricity---which can further limit an adversary.

Motivated by the lack of practical Sybil detection tools, we design and
implement heuristics, leveraging that Sybils (\emph{i}) frequently go online and
offline simultaneously, (\emph{ii}) share similarities in their configuration,
and (\emph{iii}) may change their identity fingerprint---a relay's fingerprint
is the hash over its public key---frequently, to manipulate Tor's DHT.  We
implemented these heuristics in a tool, \sys, which we then used to analyze
archived network data, dating back to 2007, to discover past attacks and
anomalies.  Finally, we characterize the Sybil groups we discover.  To sum up,
we make the following key contributions:
\begin{itemize}
	\item We design and implement a tool---\sys---to analyze past and future Tor
		network data.  While we designed it specifically for the use in Tor, our
		techniques are general in nature and can easily be applied on other
		distributed systems.
	\item We expose and characterize Sybil groups and publish them as dataset to
		stimulate future research.  We find that many different kinds of
		operators run Sybil groups, including financially-motiviated attackers,
		researchers, and unskilled troublemakers.
\end{itemize}

% Structure of the paper.
The rest of this paper is structured as follows.  We begin by discussing
related work in Section~\ref{sec:related_work}.  Section~\ref{sec:design}
presents the design of our analysis tools, which is then followed by
experimental results in Section~\ref{sec:results}.  We discuss our results in
Section~\ref{sec:discussion} and conclude the paper in
Section~\ref{sec:conclusion}.
