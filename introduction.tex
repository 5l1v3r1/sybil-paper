\section{Introduction}
\label{sec:introduction}

% Short introduction to Sybil attacks.
Distributed systems are believed to be more robust than their centralized
counterparts against attacks such as coercion, denial of service, and
surveillance.  The Tor network is no exception.  As its distributed nature is
improved, its resistance to attackers targeting choke points strenghtens.
However, the absence of a central, controlling authority opens the gates to
\emph{Sybil attacks}.  Sybil attacks---introduced by Douceur in
2002~\cite{Douceur2002a}---are defined as one physical entity controlling many
virtual entities in a distributed system.  The attacks is problematic because
many distributed systems make assumptions on the maximum number of malicious
nodes they can endure.

% How do Sybil attacks relate to Tor? -- DHT manipulation.
The potential gain of performing a Sybil attack differs from system to system.
In the case of Tor, attackers are frequently motivated by manipulating Tor's
distributed hash table (DHT)~\cite{rendspec}.  To understand how this is an
issue, we need to take a step back.  Tor clients need a hidden service
descriptor to learn how to connect to a given hidden service such as
\url{3g2upl4pq6kufc4m.onion}, the DuckDuckGo search engine.  These descriptors
are stored in Tor's DHT.  The DHT implementation suffers from a weakness that
allows an attacker to predict at which position in the DHT a descriptor will be
stored.  The attacker can then change its fingerprints to become responsible
for a given hidden service~\cite{Biryukov2013a}.\footnote{This boils down to
brute-forcing the RSA key pair of a relay until its fingerprint is reasonably
close to the descriptor's position in the DHT.}  Six Sybil relays are
sufficient to become the sole responsible party for a hidden service, allowing
the attacker to monitor (still anonymous) clients requests, or refuse to serve
the hidden service, effectively taking it down.

% How do Sybil attacks relate to Tor? -- Increased traffic exposure.
In addition to DHT manipulation, Sybils allow an attacker to increase its
exposure to traffic.  Note that generally, instead of running $n$ relays, an
attacker can run a single relay that provides just as much bandwidth as $n$
relays together.  But at some point, an attacker will have to scale
horizontally because there are limits to how much traffic can be relayed by a
single machine.  Once an attacker can see a non-trivial fraction of Tor
traffic, it is able to:
\begin{description}
	\item[Manipulate exit traffic] to steal passwords, break into TLS
		connections, or inject data~\cite{Winter2014a}.
	\item[End-to-end correlate] traffic by running entry guards as well as exit
		relays.  Note that a network-level adversary such as an ISP or a
		government might only need to run one of both.
	\item[Harvest bridge addresses] by running middle nodes and isolating the
		IP addresses of incoming Tor connections that don't originate from
		(publicly known) guard relays~\cite{Ling2012a}.
\end{description}

% Sybils can be a side effect.
Instead of being a direct means to an end, Sybils can be a \emph{side effect}
of another issue.  In Section~\ref{sec:sybil_groups}, we provide evidence for
what we believe is a botnet whose members are running Tor relays.  Similarly,
in August 2013, attackers infected computers with malware that used a Tor
hidden service as command and control server~\cite{Hopper2014a}.  In this case,
however, the infected machines were only Tor clients, and not relays.

% Why previously proposed solutions don't work.
Having outlined the problem, it is tempting to look for solutions in the
substantional amount of previous work that proposed Sybil mitigation
techniques.  In his seminal 2002 paper~\cite{Douceur2002a}, Douceur formally
showed that the only method guaranteed to keep out Sybils is a central
authority that verifies new nodes as they join the distributed system.  This
conflicts with Tor's design that seeks to eliminate central points.  In the
absence of a central authority, researchers have proposed several techniques
that boil down to identifying resources that are difficult for an attacker to
scale.  Past work has focused extensively on computational resources and edges
in social graphs, which can both be difficult to scale for an attacker.
Computational resources can be useful if an attacker that seeks to operate 100
Sybils suddently needs 100 times the computational resources she would need for
a single virtual identity.  Social graph-based Sybil detection techniques
assume that it is difficult for Sybils to establish a large number of
connections with non-Sybils.  Unfortunately, both resources do not apply to the
Tor network. Requiring relay operators to complete proof-of-works is pointless
as running a relay already involves computational work.  Social graph-based
defenses do not apply because there is no trust relationship between Tor
relays.\footnote{To clarify, relay operators are encouraged to set the
\texttt{MyFamily} option for relays under their control, but this type of
trust relationship does not span to relays run by others.}

% Our contributions.
Given the nonapplicability of existing approaches, we focus on techniques to
find Sybils in the Tor network.  We implemented these techniques in a tool,
sybilhunter, which we then use to analyze historical network data, dating back
to as early as 2007, to discover past attacks and anomalies.  Finally, we
extensively characterize the Sybil groups we discover.  To sum up, we provide
the following contributions:
\begin{itemize}
	\item We consider Sybils in Tor and show how this problem differs from
		related work.
	\item Design and implement a platform for continuous monitoring of archived
		Tor data.
	\item Analze past incidents in-depth.
\end{itemize}

% Structure of the paper.
The rest of this paper is structured as follows.  We begin by discussing
related work in Section~\ref{sec:related_work}.  Section~\ref{sec:design}
presents the design of our analysis tools, which is then followed by
experimental results in Section~\ref{sec:results}.  We discuss our results in
Section~\ref{sec:discussion} and conclude the paper in
Section~\ref{sec:conclusion}.
