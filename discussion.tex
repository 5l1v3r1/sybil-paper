\section{Discussion}
\label{sec:discussion}

\subsection{Balancing transparency and secrecy}
\label{sec:secrecy}
During the development of \sys, we pondered what a reasonable balance
between transparency and secrecy should look like in an operational setting.
While we were free to explore the choice of \sys parameters in this
paper, parameters might have to be kept secret in an operational setting.  We
want our system design and code to be open, to stimulate scientific progress,
but a freely available implementation helps attackers evade our system because
they can first test and refine their attacks offline.

To mitigate this problem, we divide our system into the \emph{open} analysis
framework and its \emph{secret} parameters.  After all, the analysis framework
is of primary interest to other researchers, whereas its parameters are mere
operational details.  Note that the authors of exitmap follow a similar
philosophy by making available exitmap's scanning framework~\cite{exitmap}, but
sharing its modules only privately.  This differentiation seems to be
sustainable as attackers are primarily interested in scanning modules, e.g.,
which URLs, protocols, and ports are probed.  The scanning framework, which is
of primary interest to the scientific community, is available as free software.

Unfortunately, we are unable to achieve a setting analogous to Kerckhoffs'
principle in cryptography, stating that a system must be secure even if all
except the key is known.  In \sys, unlike modern cryptograhpic systems,
there is a clear relationship between the input (the system parameters) and the
output (an alert).  Still, an attacker has to make an effort to stay under the
radar of our system.  Given that it is impossible to always find all Sybils, our
efforts are ultimately limited to increasing the cost of an attack, e.g.,
forcing an attacker to infer our system's parameters.

\subsection{Limitations}
\label{sec:limitations}
In Section~\ref{sec:threat_model} we argued that we are unable to prevent all
Sybil attacks.  An adversary unconstrained by time and money can add an
unlimited number of Sybils to the network.  Indeed, Table~\ref{tab:sybils}
contains six Sybil groups that \sys was unable to detect.  Exitmap, however, was
able to expose these Sybils, which emphasizes the importance of having diverse
and complementary analysis techniques to raise the bar for successful attacks.
By characterizing past attacks and documenting the evolution of recurring
attacks, we can adapt our techniques, allowing for the bar to be raised even
further.  However, this arms race is unlikely to end, barring fundamental
changes in how Tor relays are operated.  Given that attackers can stay under our
radar, our results represent a lower bound because we might have missed Sybil
groups.

Finally, \sys is unable to ascertain the purpose of a Sybil attack.  While the
purpose is frequently obvious, Table~\ref{tab:sybils} contains several Sybil
groups that we could not classify.  In such cases, it is difficult for The Tor
Project to make a call and decide if Sybils should be removed from the network.
Keeping them runs the risk of exposing users to an unknown attack, but removing
them deprives the network of bandwidth.  Often, additional context is helpful in
making a call.  For example, Sybils that are (\emph{i}) operated in
``bulletproof'' ASes~\cite{Konte2015a}, (\emph{ii}) show signs of not running
the Tor reference implementation, or (\emph{iii}) spoof information in their
router descriptor all suggest malicious intent.  In the end, Sybil groups have
to be evaluated case by case, and the advantages and disadvantages of blocking
them have to be considered.
