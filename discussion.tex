\section{Discussion}
\label{sec:discussion}
Having used \sys in practice for several months, we now elaborate on both our
operational experience and the shortcomings we encountered.

\subsection{Operational experience}
\label{sec:operational}
Our practical work with \sys taught us that analyzing Sybils frequently
requires manual verification, e.g., (\emph{i}) comparing an emerging Sybil
group with a previously disclosed one, (\emph{ii}) using exitmap to send decoy
traffic over Sybils, or (\emph{iii}) sorting and comparing information in relay
descriptors.  We found that the amount of manual work greatly depends on the
Sybils under investigation.  The MitM groups in Table~\ref{tab:sybils} were
straightforward to spot---in a matter of minutes---while the botnets required a
few hours of effort.  It is difficult to predict all analysis scenarios that
might arise in the future, so we designed \sys to be interoperable with Unix
command line tools~\cite{Pike1983a}.  \Sys's CSV-formatted output can easily be
piped into tools such as sed, awk, and grep.  We found that compact text output
was significantly easier to process, both for plotting and for manual analysis.
Aside from Sybil detection, \sys can serve as valuable tool to better
understand the Tor network and monitor its reliability.  Our techniques have
disclosed network consensus issues and can illustrate the diversity of Tor
relays, providing empirical data that can support future network design
decisions.

A key issue in the arms race of eliminating harmful relays lies in
\emph{information asymmetry}.  Our detection techniques and code are freely
available while our adversaries operate behind closed doors, creating an uphill
battle that is difficult to sustain given our limited resources.  In practice,
we can reduce this asymmetry and limit our adversaries' knowledge by keeping
secret \sys's thresholds and exitmap's detection modules, so our adversary is
left guessing what our tools seek to detect.  This differentiation between an
\emph{open} analysis framework such as the one we discuss in this paper, and
\emph{secret} configuration parameters seems to be a sustainable trade-off.

We are working with The Tor Project on incorporating our techniques in Tor
Metrics~\cite{metrics}, a web site containing network visualizations that are
frequented by numerous volunteers.  Many of these volunteers discover anomalies
and report them to The Tor Project.  By incorporating our techniques, we hope
to benefit from ``crowd-sourced'' Sybil detection.

\subsection{Limitations}
\label{sec:limitations}
In Section~\ref{sec:threat_model}, we argued that we are unable to expose all
Sybil attacks, so our results represent a lower bound.  An adversary
unconstrained by time and money can add an unlimited number of Sybils to the
network.  Indeed, Table~\ref{tab:sybils} contains six Sybil groups that \sys
was unable to detect.  Exitmap, however, was able to expose these Sybils, which
emphasizes the importance of diverse and complementary analysis techniques to
raise the bar for adversaries.  By characterizing past attacks and documenting
the evolution of recurring attacks, we can adapt our techniques, allowing for
the bar to be raised farther.  In particular, adversaries will have to spend
more time for their Sybils to remain undetected, but it is difficult to
quantify the exact cost we impose.  This arms race is unlikely to end, barring
fundamental changes in how Tor relays are operated.  

\Sys is unable to ascertain the purpose of a Sybil attack.  While the purpose
is frequently obvious, Table~\ref{tab:sybils} contains several Sybil groups
that we could not classify.  In such cases, it is difficult for The Tor Project
to make a call and decide if Sybils should be removed from the network.
Keeping them runs the risk of exposing users to an unknown attack, but removing
them deprives the network of bandwidth.  Often, additional context is helpful
in making a call.  For example, Sybils that are (\emph{i}) operated in
``bulletproof'' autonomous systems~\cite{Konte2015a}, (\emph{ii}) show signs of
not running the Tor reference implementation, or (\emph{iii}) spoof information
in their router descriptor all suggest malicious intent.  In the end, Sybil
groups have to be evaluated case by case, and the advantages and disadvantages
of blocking them have to be considered.

Finally, there is significant room for improving our nearest neighbor ranking.
For simplicity, our algorithm represents relays as strings, ignoring a wealth
nuances such as topological proximity of IP addresses, or predictable patterns
in port numbers.
