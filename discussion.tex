\section{Discussion}
\label{sec:discussion}

\subsection{Operational experience}
\label{sec:operational}
Our practical work with \sys taught us that detecting Sybils frequently requires
manual work; for example, comparing a new Sybil group with a previously
disclosed one, sending decoy traffic over Sybils, or sorting and comparing
information in their descriptors.  It is difficult to predict all kinds of
manual analyses that might be necessary in the future, which is why we designed
\sys to be highly interoperable with Unix command line tools.  Its CSV-formatted
output can easily be piped into tools such as sed, awk, and grep.  We found that
compact text output was significantly easier to deal with, both for plotting
results and for manual analysis.  We also found that \sys can serve as valuable
tool to better understand the Tor network and monitor its reliability.  Our
techniques can disclose network consensus issues and illustrate the wide
diversity of Tor relays.

We are also working with The Tor Project on incorporating our techniques in
\url{metrics.torproject.org}, a web site that contains network visualizations,
which are frequented by numerous volunteers that sometimes report anomalies.
By adding our techniques, we hope to benefit from ``crowd-sourced'' Sybil
detection.

\subsection{Limitations}
\label{sec:limitations}
In Section~\ref{sec:threat_model} we argued that we are unable to prevent all
Sybil attacks.  An adversary unconstrained by time and money can add an
unlimited number of Sybils to the network.  Indeed, Table~\ref{tab:sybils}
contains six Sybil groups that \sys was unable to detect.  Exitmap, however, was
able to expose these Sybils, which emphasizes the importance of having diverse
and complementary analysis techniques to raise the bar for successful attacks.
By characterizing past attacks and documenting the evolution of recurring
attacks, we can adapt our techniques, allowing for the bar to be raised even
further.  However, this arms race is unlikely to end, barring fundamental
changes in how Tor relays are operated.  Given that attackers can stay under our
radar, our results represent a lower bound because we might have missed Sybil
groups.

Finally, \sys is unable to ascertain the purpose of a Sybil attack.  While the
purpose is frequently obvious, Table~\ref{tab:sybils} contains several Sybil
groups that we could not classify.  In such cases, it is difficult for The Tor
Project to make a call and decide if Sybils should be removed from the network.
Keeping them runs the risk of exposing users to an unknown attack, but removing
them deprives the network of bandwidth.  Often, additional context is helpful in
making a call.  For example, Sybils that are (\emph{i}) operated in
``bulletproof'' ASes~\cite{Konte2015a}, (\emph{ii}) show signs of not running
the Tor reference implementation, or (\emph{iii}) spoof information in their
router descriptor all suggest malicious intent.  In the end, Sybil groups have
to be evaluated case by case, and the advantages and disadvantages of blocking
them have to be considered.
