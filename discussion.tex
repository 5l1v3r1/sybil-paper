\section{Discussion}
\label{sec:discussion}

\subsection{Operational experience}
\label{sec:operational}
\begin{itemize}
	\item We are running our system since YYYY-MM-DD.
	\item Difficult to get relays blacklisted because of overloaded dirauth operators.
	\item Root cause analysis difficult.  Especially if relays behave exactly
		like benign relays would.  Weighing up up pros vs. cons when
		considering to reject a relay is necessary.
\end{itemize}

\subsection{Balancing transparency and secrecy}
\label{sec:secrecy}
During the development of sybilhunter, we pondered what a reasonable balance
between transparency and secrecy could look like.  On the one hand, we want our
system design to be open to stimulate scientific progress.  On the other hand,
a freely available implementation helps attackers evade the system because they
can first test their attacks in an offline setting.

It seems difficult to achieve a setting analogous to Kerckhoffs' principle in
cryptography, which states that a system has to be secure even if all about it
is known---except the key.  There is no key in our setting.  We can, however,
divide our system into the \emph{open} analysis framework and its \emph{secret}
parameters.  It is the analysis framework that is of primary interest to other
researchers, whereas its precise parameters are a mere operational details.

Note that the authors of exitmap follow a similar philosophy by making
available exitmap's scanning framework~\cite{exitmap}, but sharing its
(typically straightforward) modules only privately.  This differentiation seems
to work well as attackers are only interested in scanning modules, e.g., which
URLs, protocols, and ports are probed.

\subsection{Limiting factors}
We mentioned in Section~\ref{sec:threat_model} that we are unable to prevent
all Sybil attacks.  An adversary not constrained by time and money will always
be able to inject Sybils.  This is already known since Douceur showed in 2002
that the only way to prevent Sybil attacks is a central authority that verifies
network participants~\cite{Douceur2002a}.  A central authority is unlikely to
be viable for the Tor network.  It would be in conflict with Tor's goal of
distributing trust and alienate relay operators.

We will now explore how costly a Sybil attack is in practice.  First, an
adversary needs a number of systems to run Tor relays on.  These systems should
be geographically distributed to maximize IP address diversity.  One option is
to rent virtual private systems, starting at around \$3 per month for 1 Gbps.
In 2011, the price for 1,000 compromised systems to install malware on (so
called \emph{loads}) ranged from \$13 (in Asia) to \$125 (in the U.S.)\cite[\S
5]{Stone-Gross2011a}.

\mynote{Analysis not yet done.}

\subsection{Are cloud providers worth the trouble?}
\begin{itemize}
	\item How much bandwidth was provided by benign, cloud-hosted relays?
	\item Are we better off blacklisting Google Cloud et al.'s IP address
		space?
	\item Tor Cloud used to use Amazon, but no longer active and bridges will
		vanish eventually.
	\item Found 71 bridges with nickname ``ec2bridge'' in bridge status document
		2015-08-22 00:54:10.  This is likely a lower bound.
\end{itemize}
